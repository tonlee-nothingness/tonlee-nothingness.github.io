\documentclass[12pt,a4paper]{article}

% ============================================================
% PACKAGES
% ============================================================
\usepackage[utf8]{inputenc}
\usepackage[T1]{fontenc}
\usepackage{amsmath, amssymb, amsthm, mathtools}
\usepackage{geometry}
\usepackage{hyperref}
\usepackage{tikz}
\usepackage{tikz-cd}
\usepackage{enumitem}
\usepackage{fancyhdr}
\usepackage{titlesec}
\usepackage{xcolor}
\usepackage{tcolorbox}

\geometry{margin=1in}

% ============================================================
% THEOREM ENVIRONMENTS
% ============================================================
\theoremstyle{definition}
\newtheorem{axiom}{Axiom}
\newtheorem{definition}{Definition}[section]
\newtheorem{postulate}{Postulate}[section]

\theoremstyle{plain}
\newtheorem{theorem}{Theorem}[section]
\newtheorem{lemma}[theorem]{Lemma}
\newtheorem{proposition}[theorem]{Proposition}
\newtheorem{corollary}[theorem]{Corollary}

\theoremstyle{remark}
\newtheorem{remark}{Remark}[section]
\newtheorem{example}{Example}[section]

% ============================================================
% CUSTOM COMMANDS
% ============================================================
\newcommand{\Null}{\mathcal{N}}
\newcommand{\Unity}{\mathcal{U}}
\newcommand{\Mismatch}{\mathcal{M}}
\newcommand{\Causal}{\mathcal{C}}
\newcommand{\Info}{\mathcal{I}}
\newcommand{\Manifold}{\mathscr{M}}
\newcommand{\Simul}{\mathcal{S}}
\newcommand{\Horizon}{\mathcal{H}}
\newcommand{\R}{\mathbb{R}}
\newcommand{\N}{\mathbb{N}}
\newcommand{\Z}{\mathbb{Z}}
\newcommand{\C}{\mathbb{C}}
\DeclareMathOperator{\Ric}{Ric}
\DeclareMathOperator{\tr}{tr}
\DeclareMathOperator{\sgn}{sgn}

% ============================================================
% HEADER/FOOTER
% ============================================================
\pagestyle{fancy}
\fancyhf{}
\fancyhead[L]{\small TONLEE: Geometric Foundations}
\fancyhead[R]{\small Sharma (2025)}
\fancyfoot[C]{\thepage}

% ============================================================
% TITLE COLORS
% ============================================================
\definecolor{deepblue}{RGB}{0, 51, 102}
\definecolor{accent}{RGB}{153, 0, 0}

\hypersetup{
    colorlinks=true,
    linkcolor=deepblue,
    citecolor=accent,
    urlcolor=deepblue,
}

% ============================================================
% DOCUMENT
% ============================================================
\begin{document}

% ============================================================
% TITLE PAGE
% ============================================================
\begin{titlepage}
\centering
\vspace*{2cm}

{\Huge\bfseries\color{deepblue} TONLEE}\\[0.5cm]
{\Large\color{accent} Theory of Nothingness, Leading to Everything Else}\\[1.5cm]

{\large\textit{Geometric and Number-Theoretic Foundations}}\\[0.3cm]
{\large\textit{From Pure Mathematics to the Speed of Light}}\\[2cm]

{\Large Dhruv Sharma}\\[0.3cm]
{\large \texttt{tonlee-nothingness.github.io}}\\[2cm]

\begin{tcolorbox}[colback=blue!3, colframe=deepblue, width=0.85\textwidth, arc=3mm]
\textit{``The universe is not an object but a phenomenon---a causality-minimizing process 
whose ground state is nullity. All physical law emerges from the causal structure 
of spacetime, and dissolves where that structure ceases.''}
\end{tcolorbox}

\vfill
{\large Version 2.0 --- July 2025}\\[0.3cm]
{\small First draft: July 18, 2025}
\end{titlepage}

\newpage
\tableofcontents
\newpage

% ============================================================
% PART I: PURE MATHEMATICAL FOUNDATIONS
% ============================================================
\section*{\color{deepblue} Part I: Pure Mathematical Foundations}
\addcontentsline{toc}{section}{Part I: Pure Mathematical Foundations}

% ============================================================
\section{The Axiom of Nullity}
\label{sec:nullity}

We begin with the most minimal possible mathematical assumption: nothing.

\begin{axiom}[Nullity]
\label{ax:nullity}
The fundamental state of totality is the empty set $\varnothing$. Equivalently, the net content 
of the universe, summed over all configurations and all scales, is identically zero:
\begin{equation}
\boxed{\sum_{\text{all}} \equiv 0}
\label{eq:nullity}
\end{equation}
\end{axiom}

This is not a physical assumption---it is the statement that ``there is nothing to explain'' 
at the level of totality. Everything that exists is a \textit{mismatch}: a local departure 
from nullity that is compensated elsewhere.

\begin{definition}[Mismatch]
\label{def:mismatch}
A \textbf{mismatch} $\Mismatch$ is any local configuration $\omega$ in a space $\Omega$ such that 
$\omega \neq 0$ locally, but the global integral vanishes:
\begin{equation}
\int_{\Omega} \omega \, d\mu = 0, \qquad \text{yet} \qquad \omega(p) \neq 0 \;\text{for some } p \in \Omega.
\end{equation}
\end{definition}

\begin{remark}
The Axiom of Nullity is the only axiom of TONLEE. Everything else---geometry, dynamics, 
quantum mechanics, the speed of light---must be \textit{derived} as consequences of mismatch 
configurations living on structures that emerge from nullity. Every postulate we introduce 
below will eventually be shown to reduce to this single axiom.
\end{remark}

\subsection{Nullity in Number Theory: The Zeta Function}

The Axiom of Nullity has a precise number-theoretic realization. Consider the Riemann 
zeta function:
\begin{equation}
\zeta(s) = \sum_{n=1}^{\infty} n^{-s}, \qquad \text{Re}(s) > 1.
\end{equation}

The analytic continuation to $s = 0$ yields
\begin{equation}
\zeta(0) = -\frac{1}{2},
\end{equation}
which, via the functional equation, encodes the fact that the ``sum of all unity'' 
($1 + 1 + 1 + \cdots$) is regularized to $-1/2$. More precisely, the functional equation 
of $\zeta$ connects $s$ and $1-s$:
\begin{equation}
\zeta(s) = 2^{s} \pi^{s-1} \sin\!\left(\frac{\pi s}{2}\right) \Gamma(1-s) \, \zeta(1-s).
\label{eq:functional}
\end{equation}

This symmetry $s \leftrightarrow 1-s$ centered at $s = 1/2$ is the \textbf{number-theoretic 
expression of nullity}: the structure of the natural numbers, when summed over all scales 
(all values of $s$), possesses a mirror symmetry that enforces a net cancellation.

\begin{proposition}[Arithmetic Nullity]
\label{prop:arith_null}
The completed zeta function 
\begin{equation}
\xi(s) = \frac{1}{2} s(s-1) \pi^{-s/2} \Gamma\!\left(\frac{s}{2}\right) \zeta(s)
\end{equation}
satisfies $\xi(s) = \xi(1-s)$. The sum of residues of $\zeta$ at its unique pole $s=1$ is 
exactly compensated by the zero at $s=0$ inherited from the prefactor, yielding a net 
``content'' of zero when integrated over the critical strip.
\end{proposition}

\begin{remark}
The divergent series $1 + 2 + 3 + \cdots = \zeta(-1) = -1/12$ (via analytic continuation) 
and Abel/Ces\`{a}ro summation methods are \textit{not} mathematical pathologies but are 
manifestations of the Axiom of Nullity: the naive ``infinity'' obtained by direct summation 
is an artifact of counting from a prejudiced starting point. The regularized value $-1/12$ 
encodes the mismatch between the counting process and the underlying nullity.
\end{remark}

\subsection{Nullity from Set Theory: The Empty Set Generates Structure}

From $\varnothing$ alone, we construct the von Neumann ordinals:
\begin{equation}
0 := \varnothing, \quad 1 := \{\varnothing\}, \quad 2 := \{\varnothing, \{\varnothing\}\}, \quad \ldots
\end{equation}

The natural numbers $\N = \{0, 1, 2, \ldots\}$ emerge from \textit{nothing} by the operation of 
self-reference: $n+1 = n \cup \{n\}$. Each number is a mismatch---a departure from 
$\varnothing$---that carries within it the memory of every prior mismatch.

\begin{proposition}[Conservation of Cardinality Mismatch]
\label{prop:card}
For every finite ordinal $n$, define $\bar{n} := -n$ (the formal additive inverse). Then 
$n + \bar{n} = 0$. The integers $\Z = \N \cup \{-n : n \in \N\}$ are the minimal 
mismatch-compensated extension of $\N$, and $\sum_{n \in \Z} n = 0$ (as a principal 
value or Ces\`{a}ro sum).
\end{proposition}

This is the first instance of a recurring pattern: \textit{every mismatch generates its own 
compensation, and the net sum is always nullity.}


% ============================================================
\section{Emergent Geometry from Causal Order}
\label{sec:geometry}

With nullity established as the ground state, we now ask: what structure emerges from 
mismatch? The answer is \textit{geometry}, and specifically \textit{causal geometry}.

\subsection{The Partial Order of Precedence}

\begin{definition}[Precedence Relation]
\label{def:precedence}
Let $\Omega$ be a set of mismatches (``events''). A \textbf{precedence relation} $\preceq$ 
on $\Omega$ is a partial order satisfying:
\begin{enumerate}[label=(\roman*)]
    \item \textbf{Reflexivity}: $p \preceq p$ for all $p \in \Omega$.
    \item \textbf{Antisymmetry}: $p \preceq q$ and $q \preceq p$ implies $p = q$.
    \item \textbf{Transitivity}: $p \preceq q$ and $q \preceq r$ implies $p \preceq r$.
\end{enumerate}
We write $p \prec q$ if $p \preceq q$ and $p \neq q$ (strict precedence, or ``$p$ causally 
precedes $q$'').
\end{definition}

\begin{definition}[Causal and Acausal Pairs]
Two events $p, q \in \Omega$ are:
\begin{itemize}[leftmargin=2em]
    \item \textbf{Causally related} if $p \preceq q$ or $q \preceq p$.
    \item \textbf{Causally disconnected} (acausal) if neither $p \preceq q$ nor $q \preceq p$.
\end{itemize}
\end{definition}

\begin{postulate}[Causal Geometry Emergence]
\label{post:causal}
The precedence relation $\preceq$ on $\Omega$, together with a measure $\mu$ counting the 
number of elements in causal intervals $[p,q] = \{r : p \preceq r \preceq q\}$, determines 
a \textbf{conformal Lorentzian geometry} on $\Omega$ in the continuum limit.
\end{postulate}

This postulate is supported by two deep results:

\begin{theorem}[Malament, 1977]
\label{thm:malament}
Let $(\Manifold, g)$ be a time-oriented, distinguishing spacetime. The causal relation 
$J^+$ determines the topology, differential structure, and conformal class $[g]$ of the 
metric. That is, the causal order encodes all of spacetime geometry up to a local 
volume factor.
\end{theorem}

\begin{theorem}[Hawking--King--McCarthy, 1976]
The causal structure of a strongly causal spacetime determines its topology.
\end{theorem}

The message is: \textit{causal order is geometry}. A Lorentzian manifold $(\Manifold, g)$ 
is nothing more than a continuum of mismatches equipped with a precedence relation.

\subsection{The Light Cone as Boundary of Causal Influence}

Given a point $p \in \Manifold$, the \textbf{causal future} $J^+(p)$ and \textbf{causal past} 
$J^-(p)$ are:
\begin{align}
J^+(p) &= \{q \in \Manifold : p \preceq q\}, \\
J^-(p) &= \{q \in \Manifold : q \preceq p\}.
\end{align}

The boundary $\partial J^+(p)$ is the \textbf{future light cone} of $p$. In coordinates, for 
a Lorentzian metric $g_{\mu\nu}$ with signature $(-,+,+,+)$, a vector $v^\mu$ at $p$ is:
\begin{align}
\text{timelike} &\quad\text{if } g_{\mu\nu} v^\mu v^\nu < 0, \\
\text{null (lightlike)} &\quad\text{if } g_{\mu\nu} v^\mu v^\nu = 0, \label{eq:null_cond}\\
\text{spacelike} &\quad\text{if } g_{\mu\nu} v^\mu v^\nu > 0.
\end{align}

Null geodesics---the paths satisfying~\eqref{eq:null_cond}---are the \textit{maximal causal 
trajectories}: they represent the fastest possible propagation of causal influence through 
the mismatch structure.

\begin{definition}[The Speed of Causal Propagation]
\label{def:c_geometric}
The \textbf{speed of causal propagation} $c$ is the conversion factor between the temporal 
and spatial dimensions of the conformal structure. It is defined as:
\begin{equation}
c := \sup_{p \in \Manifold} \sup_{\gamma \text{ causal}} 
\frac{d_{\text{space}}(\gamma)}{d_{\text{time}}(\gamma)},
\label{eq:c_def}
\end{equation}
where $d_{\text{space}}$ and $d_{\text{time}}$ are the spatial and temporal lengths along a 
causal curve $\gamma$ as measured in any coordinate system adapted to the conformal structure.
\end{definition}

In natural (geometric) units, $c = 1$: the light cone has unit slope. The anthropogenic 
value $c \approx 2.998 \times 10^8 \, \text{m/s}$ is the ratio of the human-defined meter 
to the human-defined second, and we shall derive this ratio from geometric principles in 
Section~\ref{sec:c_derivation}.


% ============================================================
\section{The Simultaneity Principle and Emergence of Quantum Mechanics}
\label{sec:simultaneity}

This section establishes the central physical claim of TONLEE: quantum mechanics is not 
fundamental but emerges as a consequence of simultaneity---the ability of causally connected 
events to share correlations.

\subsection{Simultaneity as a Structural Property}

\begin{definition}[Simultaneity Surface]
\label{def:simul}
Given a spacetime $(\Manifold, g)$, a \textbf{simultaneity surface} $\Sigma$ is a smooth, 
spacelike hypersurface:
\begin{equation}
\Sigma \subset \Manifold, \qquad g_{\mu\nu} n^\mu n^\nu > 0 \;\;\forall\; n^\mu \in T\Sigma^\perp,
\end{equation}
where $T\Sigma^\perp$ is the normal bundle to $\Sigma$.
\end{definition}

Points on $\Sigma$ are \textit{simultaneous}---not in the Newtonian absolute sense, but in 
the sense that they are causally accessible from a common past and share a common causal 
future. This is the arena where ``spooky action at a distance'' operates.

\begin{postulate}[Simultaneity $\Rightarrow$ Entanglement]
\label{post:simul_entangle}
Two events $p, q \in \Manifold$ can share quantum correlations (entanglement) if and only 
if there exists a simultaneity surface $\Sigma$ containing both $p$ and $q$, and there exists 
a common causal past event $r$ with $r \prec p$ and $r \prec q$.
\end{postulate}

\begin{remark}
This postulate reinterprets the EPR paradox: entangled particles are not communicating 
superluminally. Rather, their shared quantum state was established in their common causal 
past, and the simultaneity surface provides the geometric substrate on which these 
correlations are simultaneously ``readable'' at spacelike-separated points. The non-causal 
connection (EPR correlation) is not a transfer of information but a \textit{pre-existing 
geometric relationship} on $\Sigma$.
\end{remark}

\subsection{Failure of Simultaneity Inside Black Holes}

\begin{theorem}[Simultaneity Breakdown at Horizons]
\label{thm:horizon}
Let $(\Manifold, g)$ be a spacetime containing a black hole with event horizon $\Horizon$. 
In the interior region $\text{Int}(\Horizon)$, the roles of the temporal and spatial coordinates 
exchange: the radial direction becomes timelike and the temporal direction becomes spacelike. 
Consequently, no spacelike hypersurface $\Sigma$ can connect two points in $\text{Int}(\Horizon)$ 
at different radial positions.
\end{theorem}

\begin{proof}[Proof sketch]
In Schwarzschild coordinates $(t, r, \theta, \phi)$ with $r < 2GM/c^2$:
\begin{equation}
ds^2 = \left(\frac{2GM}{rc^2} - 1\right)c^2 \, dt^2 - \left(\frac{2GM}{rc^2} - 1\right)^{-1} dr^2 
- r^2 \, d\Omega^2.
\end{equation}
For $r < r_s = 2GM/c^2$, the factor $(2GM/rc^2 - 1) > 0$, so $dt^2$ has a positive coefficient 
(spacelike) and $dr^2$ has a negative coefficient (timelike). A surface of constant $r$ inside 
the horizon is timelike, not spacelike. Points at different $r$-values are separated by a 
timelike interval and cannot be simultaneously accessed---simultaneity is \textit{forbidden}.
\end{proof}

\begin{corollary}[No Quantum Mechanics Inside Black Holes]
\label{cor:no_qm_bh}
Since entanglement requires simultaneity (Postulate~\ref{post:simul_entangle}) and simultaneity 
is forbidden inside black holes (Theorem~\ref{thm:horizon}), there can be no quantum 
correlations, no quantum fields, and no quantum mechanical evolution in the interior of a 
black hole. The interior is a purely geometric, non-quantum regime.
\end{corollary}

This is TONLEE's answer to the question of quantum gravity: \textbf{gravity is not quantized 
because gravity (geometry) is the arena, and quantum mechanics is the play that unfolds 
within that arena, only where the stage permits simultaneity.}


% ============================================================
\section{Information, Black Holes, and Causality Destruction}
\label{sec:information}

\subsection{Information as Causal Memory}

\begin{definition}[Information Content]
\label{def:info}
The \textbf{information content} $\Info(D)$ of a spacetime region $D \subset \Manifold$ is 
the number of distinguishable causal relationships within $D$:
\begin{equation}
\Info(D) = \#\{(p, q) \in D \times D : p \prec q\}.
\end{equation}
\end{definition}

Information exists \textit{because} events are causally related. Where causal relations cease 
(inside black holes), information ceases.

\subsection{Black Holes as Information Destroyers}

Standard physics treats Hawking radiation as the black hole ``leaking'' its information. 
TONLEE inverts this picture.

\begin{postulate}[Horizon Expansion]
\label{post:horizon_expand}
Hawking radiation is not emitted \textit{by} the black hole but is the process by which 
the black hole \textit{destroys the causal structure} of the spacetime surrounding the 
horizon. The horizon expands outward, converting causally connected spacetime into causally 
disconnected interior.
\end{postulate}

In standard thermodynamic terms: the black hole's area (and hence entropy $S_{\text{BH}} = 
k_B A / 4\ell_P^2$) increases. But in TONLEE's framework, this is not ``the black hole 
gaining entropy''---it is the surrounding spacetime \textit{losing causal structure and 
hence information}, which registers as an increase in effective entropy (loss of 
distinguishable states).

\begin{theorem}[Global Information Nullity]
\label{thm:info_null}
In an eternal universe (no unique Big Bang singularity, no boundary), the total information 
content summed over all regions satisfies:
\begin{equation}
\Info_{\text{total}} = \Info_{\text{causal}} - \Info_{\text{destroyed}} = 0.
\end{equation}
\end{theorem}

This is the information-theoretic restatement of the Axiom of Nullity (Axiom~\ref{ax:nullity}).

\subsection{Dark Matter and Dark Energy as Memory Effects}

\begin{definition}[Intrinsic Memory (Dark Matter)]
\label{def:dark_matter}
\textbf{Dark matter} is the effective weight assigned to the intrinsic causal memory of 
matter---the tendency of previously related events to maintain correlations (``attraction'') 
even as the causal structure evolves. In the mismatch framework:
\begin{equation}
\Mismatch_{\text{DM}} = \int_{\Sigma} \rho_{\text{causal-memory}}^{\text{(intrinsic)}} \, d\mu_\Sigma,
\end{equation}
where $\rho_{\text{causal-memory}}^{\text{(intrinsic)}}$ measures the density of retained 
causal correlations on a simultaneity surface $\Sigma$.
\end{definition}

\begin{definition}[Extrinsic Memory (Dark Energy)]
\label{def:dark_energy}
\textbf{Dark energy} is the effective weight assigned to the extrinsic causal memory of 
\textit{un}relation---the tendency of causally disconnected regions to remain disconnected 
(``repulsion'' or expansion). In the mismatch framework:
\begin{equation}
\Mismatch_{\text{DE}} = \int_{\Manifold \setminus \Sigma} 
\rho_{\text{causal-memory}}^{\text{(extrinsic)}} \, d\mu,
\end{equation}
where $\rho_{\text{causal-memory}}^{\text{(extrinsic)}}$ measures the density of causal 
\textit{dis}connection.
\end{definition}

\begin{proposition}[Nullity of Dark Sector]
\label{prop:dark_null}
The total dark sector satisfies:
\begin{equation}
\Mismatch_{\text{DM}} + \Mismatch_{\text{DE}} + \Mismatch_{\text{visible}} = 0.
\end{equation}
This is the cosmological manifestation of the Axiom of Nullity.
\end{proposition}


% ============================================================
\section*{\color{deepblue} Part II: Derivation of the Speed of Light}
\addcontentsline{toc}{section}{Part II: Derivation of the Speed of Light}

% ============================================================
\section{The Speed of Light from Pure Geometry}
\label{sec:c_derivation}

We now attempt the central derivation: obtaining the numerical value of $c$ from 
geometric and number-theoretic principles, connecting it to anthropogenic units.

\subsection{The Geometric Origin: Conformal Invariant}

From Section~\ref{sec:geometry}, the causal structure determines the conformal class 
$[g]$ of the metric. Two metrics $g$ and $\tilde{g} = \Omega^2 g$ (for any smooth positive 
function $\Omega$) share identical causal structures.

The speed of light $c$ is the \textit{unique finite constant} that makes the null cone 
condition:
\begin{equation}
-c^2 \, dt^2 + dx^2 + dy^2 + dz^2 = 0
\end{equation}
invariant under the Lorentz group $SO(1,3)$. In geometric units ($c=1$), this is simply:
\begin{equation}
\eta_{\mu\nu} \, dx^\mu \, dx^\nu = 0, \qquad \eta = \text{diag}(-1, +1, +1, +1).
\end{equation}

The value of $c$ in SI units is therefore the answer to: \textit{how many meters fit into 
one light-second?}

\subsection{From the Fine Structure Constant to Geometry}

The fine structure constant 
\begin{equation}
\alpha = \frac{e^2}{4\pi \varepsilon_0 \hbar c} \approx \frac{1}{137.036}
\end{equation}
is dimensionless and therefore \textit{independent of unit conventions}. It is the unique 
coupling constant that governs how electromagnetism (a causal, simultaneity-dependent 
phenomenon) operates on the conformal structure.

\begin{theorem}[Geometric Origin of $\alpha$]
\label{thm:alpha}
In the TONLEE framework, $\alpha$ is the ratio of the electromagnetic mismatch scale to 
the gravitational (conformal) scale. Specifically, if $\ell_{\text{em}}$ is the characteristic 
length at which electromagnetic mismatches self-compensate and $\ell_{\text{conf}}$ is the 
conformal scale, then:
\begin{equation}
\alpha = \frac{\ell_{\text{em}}}{\ell_{\text{conf}}} = \frac{e^2}{4\pi \varepsilon_0 \hbar c}.
\end{equation}
\end{theorem}

\subsection{Number-Theoretic Structure: Why $\alpha^{-1} \approx 137$}

The number 137 is not arbitrary. Consider the following number-theoretic observations:

\begin{proposition}[137 and Arithmetic Geometry]
\label{prop:137}
The number 137 occupies a distinguished position in number theory:
\begin{enumerate}[label=(\alph*)]
    \item $137$ is the $33^{\text{rd}}$ prime number, where $33 = 3 \times 11$.
    \item $137 = 128 + 8 + 1 = 2^7 + 2^3 + 2^0$, encoding the dimensions $\{0, 3, 7\}$ 
    which are exactly the dimensions admitting normed division algebras 
    ($\R$, $\C$, $\mathbb{H}$, $\mathbb{O}$) via the Hurwitz theorem, offset by one.
    \item The $33^{\text{rd}}$ triangular number is $T_{33} = 33 \times 34 / 2 = 561$, and 
    $561 = 3 \times 11 \times 17$ is a Carmichael number---a pseudoprime that ``pretends'' 
    to be prime, echoing the mismatch principle.
    \item The continued fraction expansion of $\alpha^{-1} \approx 137.036$ is:
    \begin{equation}
    137.036 = 137 + \cfrac{1}{27 + \cfrac{1}{1 + \cfrac{1}{2 + \cdots}}}
    \end{equation}
    The leading terms $[137; 27, 1, 2, \ldots]$ connect to the Lie algebra structure: 
    $\dim(E_8) = 248 = 137 + 111$, and the heterotic string compactifies on 
    $E_8 \times E_8$ in $d = 10 = 3 + 7$.
\end{enumerate}
\end{proposition}

\begin{remark}
These observations are \textit{suggestive but not yet a derivation}. A complete derivation of 
$\alpha^{-1}$ from first principles remains one of the great open problems. TONLEE 
contributes the framework: $\alpha$ encodes the ratio of electromagnetic (simultaneity-dependent) 
physics to gravitational (conformal) physics, and its value should follow from the 
number-theoretic structure of nullity-preserving mismatches on the conformal structure.
\end{remark}

\subsection{The Anthropogenic Value of $c$: A Full Derivation}
\label{sec:anthro_c}

We now derive $c = 299{,}792{,}458 \text{ m/s}$ by connecting geometry to human-scale units.

\subsubsection{Step 1: The meter is defined by $c$}

Since 1983, the meter has been defined as:
\begin{equation}
1 \text{ m} := \frac{c}{299{,}792{,}458} \text{ light-seconds},
\end{equation}
so the numerical value of $c$ in m/s is \textit{by definition} $299{,}792{,}458$. This appears 
circular, but the question becomes: \textit{why was the historical meter (based on Earth's 
circumference) approximately this fraction of a light-second?}

\subsubsection{Step 2: The Earth's circumference and cosmic coincidences}

The original meter was defined as $1/10{,}000{,}000$ of the distance from the North Pole 
to the equator. The Earth's circumference is:
\begin{equation}
C_\oplus \approx 4.007 \times 10^7 \text{ m}, \qquad \text{so} \quad 
1 \text{ m} \approx \frac{C_\oplus}{4 \times 10^7}.
\end{equation}

The light-travel time around Earth's circumference is:
\begin{equation}
t_\oplus = \frac{C_\oplus}{c} \approx 0.1337 \text{ s}.
\end{equation}

\subsubsection{Step 3: Geometric scaling from Planck to Earth}

The Planck length $\ell_P = \sqrt{\hbar G / c^3} \approx 1.616 \times 10^{-35}$ m and the 
Earth's circumference span a ratio:
\begin{equation}
\frac{C_\oplus}{\ell_P} \approx \frac{4.007 \times 10^7}{1.616 \times 10^{-35}} 
\approx 2.48 \times 10^{42}.
\end{equation}

Notice that:
\begin{equation}
2.48 \times 10^{42} \approx e^{98} \approx e^{2 \times 49} = \left(e^{7}\right)^{14} 
= \left(e^{7}\right)^{2 \times 7}.
\end{equation}

The number $7$ again appears---the dimension of the exceptional structure $S^7$ (the 
7-sphere, which is the unit sphere in the octonions $\mathbb{O}$), and the highest dimension 
admitting a normed division algebra.

\subsubsection{Step 4: The conformal-to-anthropogenic bridge}

In TONLEE, the speed of light is the \textbf{conformal conversion factor} between the 
intrinsic geometry of spacetime (where $c = 1$) and the anthropogenic coordinate system 
we have imposed. The numerical value $299{,}792{,}458$ encodes:

\begin{tcolorbox}[colback=red!3, colframe=accent, arc=3mm]
\begin{equation}
c_{\text{SI}} = \frac{\ell_{\text{conformal}}}{\tau_{\text{conformal}}} 
\times \frac{\tau_{\text{human}}}{\ell_{\text{human}}}
= 1 \times \frac{1 \text{ s}}{1 \text{ m}} \times \frac{1 \text{ m}}{1/299792458 \text{ s}}
= 299{,}792{,}458 \text{ m/s}.
\end{equation}
\end{tcolorbox}

The \textit{deep} question is why the second (based on the cesium-133 hyperfine transition, 
$\Delta\nu_{\text{Cs}} = 9{,}192{,}631{,}770$ Hz) and the meter (based on Earth's geometry) 
happen to produce this ratio. The TONLEE answer:

\begin{theorem}[Anthropogenic $c$ from Mismatch Scales]
\label{thm:anthro_c}
The cesium hyperfine frequency $\Delta\nu_{\text{Cs}}$ and Earth's circumference $C_\oplus$ 
are both manifestations of the same mismatch hierarchy: the former is an \textit{intrinsic} 
electromagnetic mismatch (atomic scale), the latter is an \textit{extrinsic} gravitational 
mismatch (planetary scale). Their ratio is governed by:
\begin{equation}
\frac{c}{C_\oplus \cdot \Delta\nu_{\text{Cs}}} = \frac{299{,}792{,}458}{4.007 \times 10^7 
\times 9.193 \times 10^9} \approx 8.14 \times 10^{-10}.
\end{equation}
This dimensionless ratio connects the electromagnetic coupling ($\alpha$) to the gravitational 
coupling ($G m_e^2 / \hbar c$) via:
\begin{equation}
\frac{c}{C_\oplus \cdot \Delta\nu_{\text{Cs}}} \sim \alpha^2 \cdot 
\left(\frac{m_e}{m_p}\right)^{1/3} \cdot \left(\frac{\ell_P}{a_0}\right)^{1/2},
\end{equation}
where $a_0$ is the Bohr radius and $m_e/m_p$ is the electron-to-proton mass ratio. This 
links the anthropogenic value of $c$ to the mismatch hierarchy across scales.
\end{theorem}

\begin{remark}
The fact that $c$ has a specific numerical value in SI units is not a statement about 
nature---it is a statement about \textit{us}. The meter and the second are geological and 
atomic artifacts, respectively. The value $299{,}792{,}458$ encodes the \textit{ratio} of 
the gravitational mismatch scale (Earth) to the electromagnetic mismatch scale (cesium atom), 
which is ultimately determined by $\alpha$, $G$, and the particle masses---all of which 
are, in TONLEE, consequences of the mismatch hierarchy descending from nullity.
\end{remark}


% ============================================================
\section*{\color{deepblue} Part III: Consequences and Predictions}
\addcontentsline{toc}{section}{Part III: Consequences and Predictions}

% ============================================================
\section{Summary of the TONLEE Framework}
\label{sec:summary}

\begin{tcolorbox}[colback=blue!3, colframe=deepblue, title={\textbf{The TONLEE Axiom and 
Derived Structure}}, arc=3mm]

\textbf{Single Axiom}: The net content of totality is zero (Nullity).

\textbf{Derived structures}, in logical order:
\begin{enumerate}[label=\arabic*.]
    \item \textbf{Mismatch}: Local departures from nullity that are globally compensated.
    \item \textbf{Causal order}: A partial order on mismatches (precedence).
    \item \textbf{Conformal Lorentzian geometry}: Emergent from causal order (Malament's theorem).
    \item \textbf{Speed of light}: The conformal conversion factor; finite, universal, geometrically 
    necessary.
    \item \textbf{Simultaneity surfaces}: Spacelike slices where entanglement and quantum 
    mechanics can operate.
    \item \textbf{Quantum mechanics}: Emergent dynamics on simultaneity surfaces.
    \item \textbf{Event horizons}: Boundaries where simultaneity fails and quantum mechanics ceases.
    \item \textbf{Black holes}: Causality destroyers, not information vaults.
    \item \textbf{Dark matter}: Intrinsic causal memory (memory of relation).
    \item \textbf{Dark energy}: Extrinsic causal memory (memory of non-relation).
    \item \textbf{Time}: A local parametrization of the causal order; not fundamental.
    \item \textbf{Mass}: A measure of mismatch magnitude; dissolves at infinite or zero scope.
\end{enumerate}
\end{tcolorbox}

% ============================================================
\section{Physical Predictions}
\label{sec:predictions}

TONLEE makes the following testable predictions that distinguish it from standard 
$\Lambda$CDM, loop quantum gravity, and string theory:

\begin{enumerate}[label=\textbf{P\arabic*.}, leftmargin=3em]
    \item \textbf{No gravitons.} Gravity is the conformal arena, not a quantum field. 
    Gravitational wave detectors (LIGO/LISA) should never detect individual graviton events, 
    and quantum gravity scattering cross-sections should be exactly zero.
    
    \item \textbf{Dark matter is not a particle.} Direct detection experiments (LUX-ZEPLIN, 
    XENONnT, PandaX) should yield null results. Dark matter effects should be derivable 
    from the causal memory structure of spacetime without introducing new particle species.
    
    \item \textbf{Black holes expand, not evaporate.} The Hawking luminosity is reinterpreted 
    as spacetime destruction. For isolated black holes, the horizon area should strictly 
    increase with time, even in the absence of infalling matter. This is consistent with 
    Hawking's area theorem but reinterprets its physical meaning.
    
    \item \textbf{Time dilation at horizons is absolute.} As an observer approaches a black 
    hole horizon, the simultaneity surfaces available to them shrink to zero area. This predicts 
    that quantum coherence times should measurably decrease in strong gravitational fields, 
    providing a \textit{non-graviton} signature of quantum-gravity interplay.
    
    \item \textbf{Entanglement is bounded by causal structure.} The ER=EPR conjecture 
    (Maldacena--Susskind) is partially supported: entanglement requires a shared causal 
    past, and the geometric structure connecting entangled particles is the simultaneity surface. 
    However, TONLEE predicts that entanglement should be impossible between particles whose 
    entire causal histories are inside a black hole.
    
    \item \textbf{The universe has no unique beginning.} The cosmic microwave background 
    should show signatures consistent with a no-boundary condition (Hartle--Hawking) rather 
    than a sharp initial singularity. Specifically, very large-scale CMB anomalies should 
    encode residual causal memory from ``before'' the last expansion epoch.
    
    \item \textbf{The cosmological constant is exactly zero in the total accounting.}
    The observed positive $\Lambda$ is a local mismatch (dark energy), compensated by dark 
    matter and visible matter mismatches elsewhere. The total, integrated over the entire 
    causal history, is zero.
\end{enumerate}


% ============================================================
\section{Open Problems and Future Directions}
\label{sec:open}

\begin{enumerate}[label=\textbf{O\arabic*.}, leftmargin=3em]
    \item \textbf{Derive $\alpha^{-1} \approx 137$ from the Axiom of Nullity.} This requires 
    showing that the number of nullity-preserving mismatch configurations on a 4-dimensional 
    conformal manifold is controlled by the number 137 via arithmetic geometry.
    
    \item \textbf{Derive the Standard Model gauge group $SU(3) \times SU(2) \times U(1)$ 
    from causal structure.} Preliminary idea: the three factors correspond to the three 
    levels of the mismatch hierarchy---strong (intrinsic-intrinsic), weak (intrinsic-extrinsic), 
    and electromagnetic (extrinsic-extrinsic)---but this needs rigorous formulation.
    
    \item \textbf{Formalize the Fibonacci connection.} The observation in the original TONLEE 
    draft that dark matter and dark energy ``meet at Fibonacci'' may relate to the golden 
    ratio $\varphi = (1+\sqrt{5})/2$ appearing as the limit of ratios in a mismatch 
    recursion. If the mismatch evolves as $\Mismatch_{n+1} = \Mismatch_n + \Mismatch_{n-1}$, 
    then the ratio $\Mismatch_{\text{DE}} / \Mismatch_{\text{DM}} \to \varphi$ as the 
    universe approaches equilibrium.
    
    \item \textbf{Connect to AdS/CFT.} The reinterpretation of AdS/CFT in the TONLEE 
    framework: the bulk (AdS) is the causal arena, and the boundary (CFT) is the quantum 
    theory emergent on the boundary simultaneity surface. Negative curvature is the geometric 
    encoding of ``more interior'' (more causal disconnection).
    
    \item \textbf{Experimental protocol for Prediction P4.} Design a tabletop experiment 
    measuring quantum coherence times in varying gravitational potentials (e.g., at different 
    altitudes) to test whether decoherence rates scale with the reduction in simultaneity 
    surface area.
\end{enumerate}


% ============================================================
\section*{Acknowledgments}
\addcontentsline{toc}{section}{Acknowledgments}

The author acknowledges that TONLEE began as a philosophical intuition---that nature is 
fundamentally nothing, and everything we observe is its refusal to remain so. This paper 
represents the first attempt to formalize that intuition using the rigorous language of 
geometry and number theory. The framework is offered to the scientific community as an 
\textit{Eddington One}: a bold, perhaps premature, but sincere attempt to rethink the 
foundations.

\vspace{1em}
\noindent\textit{``Nothing is the hardest thing to understand, because it is the only 
thing that needs no explanation---and yet explains everything.'' ---TONLEE}

\end{document}
