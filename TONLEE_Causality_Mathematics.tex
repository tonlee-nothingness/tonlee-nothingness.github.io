\documentclass[12pt,a4paper]{article}

% ---- Packages ----
\usepackage[utf8]{inputenc}
\usepackage[T1]{fontenc}
\usepackage{amsmath,amssymb,amsthm,mathtools}
\usepackage{geometry}
\geometry{margin=1in}
\usepackage{enumitem}
\usepackage{hyperref}
\hypersetup{colorlinks=true,linkcolor=blue!70!black,citecolor=green!50!black,urlcolor=blue!60!black}
\usepackage{cleveref}
\usepackage{tikz}
\usetikzlibrary{arrows.meta,decorations.pathmorphing,calc}
\usepackage{tikz-cd}
\usepackage{thmtools}
\usepackage{xcolor}
\usepackage{fancyhdr}
\usepackage{titlesec}

% ---- Page style ----
\pagestyle{fancy}
\fancyhf{}
\fancyhead[L]{\small\textsc{TONLEE: Rigorous Causal Foundations}}
\fancyhead[R]{\small Sharma (2025--2026)}
\fancyfoot[C]{\thepage}
\renewcommand{\headrulewidth}{0.4pt}

% ---- Theorem environments ----
\theoremstyle{plain}
\newtheorem{theorem}{Theorem}[section]
\newtheorem{proposition}[theorem]{Proposition}
\newtheorem{lemma}[theorem]{Lemma}
\newtheorem{corollary}[theorem]{Corollary}

\theoremstyle{definition}
\newtheorem{definition}[theorem]{Definition}
\newtheorem{axiom}{Axiom}
\newtheorem{postulate}{Postulate}
\newtheorem{example}[theorem]{Example}

\theoremstyle{remark}
\newtheorem{remark}[theorem]{Remark}
\newtheorem{notation}[theorem]{Notation}

% ---- Custom commands ----
\DeclareMathOperator{\supp}{supp}
\DeclareMathOperator{\Tr}{Tr}
\DeclareMathOperator{\sgn}{sgn}
\DeclareMathOperator{\Ric}{Ric}
\DeclareMathOperator{\Scal}{Scal}
\DeclareMathOperator{\Vol}{Vol}
\DeclareMathOperator{\Dom}{Dom}
\DeclareMathOperator{\Int}{Int}
\newcommand{\R}{\mathbb{R}}
\newcommand{\Z}{\mathbb{Z}}
\newcommand{\N}{\mathbb{N}}
\newcommand{\C}{\mathbb{C}}
\newcommand{\Hil}{\mathcal{H}}
\newcommand{\Alg}{\mathfrak{A}}
\newcommand{\Lag}{\mathcal{L}}
\newcommand{\II}{\mathcal{I}}
\newcommand{\MM}{\mathcal{M}}

\title{%
  \textbf{TONLEE: Rigorous Mathematical Foundations}\\[6pt]
  \large Causal Structure, Variational Nullity, and Emergent Dynamics\\
  from the Conservation of Zero%
}
\author{Dhruv Sharma\\\texttt{tonlee-nothingness.github.io}}
\date{Version 3.0 \quad---\quad February 2026\\[2pt]
\footnotesize Based on TONLEE v1.0 (July 2025), Geometric Foundations v2.0 (July 2025),\\
and Formal Extensions (February 2026)}

\begin{document}
\maketitle

\begin{abstract}
We present a mathematically rigorous formulation of the TONLEE framework
(Theory of Nothingness, Leading to Everything Else). Starting from a single
axiom---the conservation of nullity, stating that the total ontological content
of the universe is identically zero---we construct the causal, geometric, and
dynamical structures of physics as derived consequences. The paper proceeds
in four stages: (I)~we formalize nullity within measure theory and functional
analysis, establishing the mismatch decomposition on signed measure spaces;
(II)~we derive Lorentzian causal geometry from a partial order on the mismatch
space, connecting to the theorems of Malament and Hawking--King--McCarthy;
(III)~we construct the ignorance functional---a variational principle whose
extremization yields the equations of motion for both the gross (mass-energy)
and subtle (quantum-informational) sectors, with the speed of light emerging
as the characteristic speed enforcing well-posedness; (IV)~we analyze the
Fibonacci generating function identity $\sum_{n=0}^{\infty} F_n x^n\big|_{x=1}^{\text{reg}} = -1$
as a regularized reflection encoding the null identity $+1 + (-1) = 0$.
All claims are stated as precise definitions, propositions, and theorems
with proofs or explicit proof strategies.
\end{abstract}

\tableofcontents
\newpage

%======================================================================
% PART I
%======================================================================
\part{The Axiom of Nullity: Measure-Theoretic Foundations}

\section{The Zero Axiom and Its Formalization}\label{sec:axiom}

\subsection{Statement of the Axiom}

\begin{axiom}[Conservation of Nullity]\label{ax:nullity}
Let $(\Omega, \Sigma, \mu)$ be a signed measure space representing the totality of
ontological content. The total measure is identically zero:
\begin{equation}\label{eq:nullity}
  \mu(\Omega) = 0.
\end{equation}
Equivalently, for every measurable decomposition $\Omega = A \sqcup A^c$:
\begin{equation}\label{eq:nullity-decomp}
  \mu(A) + \mu(A^c) = 0 \quad\Longrightarrow\quad \mu(A^c) = -\mu(A).
\end{equation}
\end{axiom}

\begin{remark}
The use of a \emph{signed} measure (rather than a positive measure) is essential:
it is precisely the existence of both positive and negative contributions that
permits $\mu(\Omega) = 0$ while allowing $\mu(A) \neq 0$ for proper subsets $A \subsetneq \Omega$.
This is the mathematical encoding of the physical intuition that
``everything is a local departure from nothing, compensated elsewhere.''
\end{remark}

\subsection{The Mismatch Space}

\begin{definition}[Mismatch]\label{def:mismatch}
A \emph{mismatch} on $(\Omega, \Sigma, \mu)$ is a measurable function
$\omega: \Omega \to \R$ such that:
\begin{enumerate}[label=(\roman*)]
  \item $\omega \in L^1(\Omega, \mu)$, i.e., $\int_\Omega |\omega|\,d|\mu| < \infty$;
  \item The global integral vanishes:
    \begin{equation}\label{eq:mismatch-global}
      \int_\Omega \omega\,d\mu = 0;
    \end{equation}
  \item There exists a set $S \in \Sigma$ with $|\mu|(S) > 0$ such that
    $\omega(p) \neq 0$ for $|\mu|$-almost every $p \in S$.
\end{enumerate}
The space of all mismatches is denoted $\MM(\Omega, \mu)$.
\end{definition}

\begin{proposition}[Vector Space Structure]\label{prop:mismatch-vec}
$\MM(\Omega, \mu)$ is a real vector subspace of $L^1(\Omega, |\mu|)$: if
$\omega_1, \omega_2 \in \MM(\Omega, \mu)$ and $\alpha, \beta \in \R$, then
$\alpha\omega_1 + \beta\omega_2 \in \MM(\Omega, \mu)$.
\end{proposition}

\begin{proof}
Linearity of the integral gives
$\int_\Omega (\alpha\omega_1 + \beta\omega_2)\,d\mu
  = \alpha\int_\Omega \omega_1\,d\mu + \beta\int_\Omega \omega_2\,d\mu = 0$.
The $L^1$ condition follows from the triangle inequality, and condition~(iii)
is preserved for generic linear combinations.
\end{proof}

\subsection{The Hahn Decomposition and Intrinsic/Extrinsic Structure}

By the Hahn decomposition theorem, there exist disjoint measurable sets
$P, N$ with $\Omega = P \sqcup N$ such that $\mu(A) \geq 0$ for all measurable
$A \subseteq P$ and $\mu(A) \leq 0$ for all measurable $A \subseteq N$. Then:
\begin{equation}
  \mu^+(A) := \mu(A \cap P), \qquad \mu^-(A) := -\mu(A \cap N),
\end{equation}
with $\mu = \mu^+ - \mu^-$ the Jordan decomposition. The nullity axiom
\eqref{eq:nullity} becomes:
\begin{equation}\label{eq:jordan-null}
  \mu^+(\Omega) = \mu^-(\Omega).
\end{equation}

\begin{definition}[Gross and Subtle Decomposition]\label{def:gross-subtle}
Given a mismatch $\omega \in \MM(\Omega, \mu)$, define the \emph{Hahn-aligned
decomposition}:
\begin{equation}
  \omega = \omega^+ - \omega^-, \qquad
  \omega^+ := \omega \cdot \mathbf{1}_P, \quad
  \omega^- := -\omega \cdot \mathbf{1}_N,
\end{equation}
where $\mathbf{1}_P, \mathbf{1}_N$ are the indicator functions of the Hahn sets.
In the TONLEE interpretation:
\begin{itemize}[leftmargin=2em]
  \item $\omega^+$ encodes the \emph{gross body} (sth\=ula \'{s}ar\={\i}ra): mass-energy configurations,
  \item $\omega^-$ encodes the \emph{subtle body} (s\=uk\d{s}ma \'{s}ar\={\i}ra): informational/quantum configurations.
\end{itemize}
The nullity axiom guarantees $\|\omega^+\|_{L^1} = \|\omega^-\|_{L^1}$ in the sense of
\eqref{eq:jordan-null}.
\end{definition}

\subsection{Number-Theoretic Realization: The Riemann Zeta Function}\label{sec:zeta}

The nullity principle finds a precise realization in the analytic structure
of the Riemann zeta function.

\begin{proposition}[Symmetry of the Completed Zeta Function]\label{prop:zeta}
Define the completed zeta function:
\begin{equation}\label{eq:xi}
  \xi(s) := \frac{1}{2}s(s-1)\pi^{-s/2}\,\Gamma\!\left(\frac{s}{2}\right)\zeta(s).
\end{equation}
Then $\xi$ is an entire function of order~$1$ satisfying the functional equation:
\begin{equation}\label{eq:xi-sym}
  \xi(s) = \xi(1 - s) \qquad \forall\, s \in \C.
\end{equation}
This reflection symmetry about the line $\Re(s) = \tfrac{1}{2}$ is a
number-theoretic instance of the nullity principle: the arithmetic content
encoded in $\zeta(s)$ for $\Re(s) > \tfrac{1}{2}$ is exactly mirrored
by the content for $\Re(s) < \tfrac{1}{2}$, yielding a net ``cancellation''
in the sense that $\xi$ is invariant under $s \mapsto 1 - s$.
\end{proposition}

\begin{proof}
This is Riemann's 1859 result. Starting from the Mellin-transform representation
\begin{equation}
  \pi^{-s/2}\Gamma\!\left(\frac{s}{2}\right)\zeta(s)
  = \int_0^\infty x^{s/2 - 1}\left(\sum_{n=1}^\infty e^{-\pi n^2 x}\right)dx
  = \int_0^\infty x^{s/2 - 1}\,\psi(x)\,dx,
\end{equation}
where $\psi(x) = \sum_{n=1}^\infty e^{-\pi n^2 x}$, one uses the Jacobi theta
function identity
\begin{equation}\label{eq:jacobi}
  1 + 2\psi(x) = \frac{1}{\sqrt{x}}\bigl(1 + 2\psi(1/x)\bigr),
\end{equation}
which itself is a consequence of Poisson summation (a Fourier-analytic
``nullity'' identity: the sum over a lattice equals the sum over the dual
lattice). Splitting the Mellin integral at $x = 1$ and applying
\eqref{eq:jacobi} to the $[0,1]$ piece yields $\xi(s) = \xi(1-s)$.
\end{proof}

\begin{remark}[Regularized Values as Mismatch Residues]\label{rem:zeta-values}
The regularized values
\begin{equation}
  \zeta(0) = -\tfrac{1}{2}, \qquad \zeta(-1) = -\tfrac{1}{12},
\end{equation}
obtained by analytic continuation, should be understood within the nullity
framework not as ``sums'' in the classical sense, but as the unique values
consistent with the functional equation \eqref{eq:xi-sym}---they are
\emph{forced} by the reflection symmetry. The formal divergence of the
partial sums $\sum_{n=1}^N 1$ and $\sum_{n=1}^N n$ reflects a failure to
account for the compensating structure encoded in $\zeta(1-s)$.
\end{remark}


%======================================================================
% PART II
%======================================================================
\part{Emergent Lorentzian Geometry from Causal Order}

\section{The Causal Set and Its Continuum Limit}\label{sec:causal-order}

\subsection{The Precedence Poset}

\begin{definition}[Causal Set (Poset of Mismatches)]\label{def:causet}
Let $(\Omega, \preceq)$ be a partially ordered set (poset) whose elements are
mismatches $p, q \in \Omega$. The partial order $\preceq$ satisfies:
\begin{enumerate}[label=(P\arabic*)]
  \item \textbf{Reflexivity:} $p \preceq p$ for all $p \in \Omega$.
  \item \textbf{Antisymmetry:} $p \preceq q$ and $q \preceq p$ implies $p = q$.
  \item \textbf{Transitivity:} $p \preceq q$ and $q \preceq r$ implies $p \preceq r$.
  \item \textbf{Local finiteness:} For all $p, q \in \Omega$, the causal interval
    $[p,q] := \{r \in \Omega : p \preceq r \preceq q\}$ is finite.
\end{enumerate}
We write $p \prec q$ if $p \preceq q$ and $p \neq q$ (strict precedence).
\end{definition}

\begin{definition}[Causal and Acausal Pairs]\label{def:causal-acausal}
Two elements $p, q \in \Omega$ are:
\begin{itemize}[leftmargin=2em]
  \item \emph{Causally related} if $p \preceq q$ or $q \preceq p$.
  \item \emph{Causally disconnected (acausal)} if neither $p \preceq q$ nor $q \preceq p$,
    denoted $p \perp q$.
\end{itemize}
\end{definition}

\begin{definition}[Causal Futures and Pasts]\label{def:J-sets}
For $p \in \Omega$:
\begin{equation}
  J^+(p) := \{q \in \Omega : p \preceq q\}, \qquad
  J^-(p) := \{q \in \Omega : q \preceq p\}.
\end{equation}
The \emph{causal interval} (or Alexandrov set) is $J(p,q) := J^+(p) \cap J^-(q)$.
\end{definition}

\subsection{Volume from Counting: The Hauptvermutung of Causal Set Theory}

The causal set program (Bombelli--Lee--Meyer--Sorkin, 1987) proposes that
the cardinality of causal intervals encodes spacetime volume:

\begin{postulate}[Volume--Number Correspondence]\label{post:volume}
In the continuum limit, the number of elements in a causal interval is
proportional to the spacetime volume of the corresponding Alexandrov set:
\begin{equation}\label{eq:volume-number}
  |[p,q]| \;\sim\; \frac{\Vol_g(J(p,q))}{\ell_{\text{fund}}^{\,n}},
\end{equation}
where $\ell_{\text{fund}}$ is the fundamental length scale and $n = \dim M$.
\end{postulate}

\subsection{Recovery of Lorentzian Geometry: The Reconstruction Theorems}

The following are precise theorems from the mathematical relativity literature
that justify the TONLEE postulate that ``causal order is geometry.''

\begin{theorem}[Hawking--King--McCarthy, 1976]\label{thm:HKM}
Let $(M, g)$ be a strongly causal spacetime. Then the manifold topology
on $M$ is determined by the causal relation $\preceq$ alone: the topology
coincides with the \emph{Alexandrov topology}, whose basis consists of sets
$I^+(p) \cap I^-(q)$ for all $p, q \in M$.
\end{theorem}

\begin{theorem}[Malament, 1977]\label{thm:Malament}
Let $(M_1, g_1)$ and $(M_2, g_2)$ be time-oriented, past- and
future-distinguishing spacetimes. If there exists a bijection
$\varphi: M_1 \to M_2$ preserving the causal relation
($p \preceq q \iff \varphi(p) \preceq \varphi(q)$),
then $\varphi$ is a conformal isometry: there exists a smooth positive
function $\Lambda: M_1 \to \R_{>0}$ such that
\begin{equation}\label{eq:malament}
  \varphi^* g_2 = \Lambda^2\, g_1.
\end{equation}
In particular, the causal order determines the manifold, its topology,
its smooth structure, and the metric up to a conformal factor.
\end{theorem}

\begin{corollary}[Causal Order $\Rightarrow$ Conformal Lorentzian Geometry]\label{cor:causal-geometry}
The pair $(\Omega, \preceq)$ from \Cref{def:causet}, in its continuum
limit supplemented by the volume data of \Cref{post:volume}, determines
a unique (up to diffeomorphism) Lorentzian manifold $(M, g)$ with:
\begin{enumerate}[label=(\roman*)]
  \item A manifold topology (by \Cref{thm:HKM}),
  \item A conformal class $[g]$ (by \Cref{thm:Malament}),
  \item A specific representative metric $g \in [g]$ (by the volume data).
\end{enumerate}
\end{corollary}

\section{Light Cone Structure and the Speed of Causal Propagation}\label{sec:lightcone}

\subsection{Tangent Space Decomposition}

Let $(M, g)$ be the emergent Lorentzian manifold from \Cref{cor:causal-geometry},
with $g$ of signature $(-,+,\ldots,+)$. At each point $p \in M$, the
tangent space $T_pM$ is partitioned by $g_p$ into:

\begin{definition}[Causal Character of Vectors]\label{def:causal-char}
A vector $v \in T_pM$ is:
\begin{alignat}{2}
  &\textit{timelike} &\quad&\text{if } g_p(v,v) < 0, \label{eq:timelike}\\
  &\textit{null (lightlike)} &\quad&\text{if } g_p(v,v) = 0,\; v \neq 0, \label{eq:null}\\
  &\textit{spacelike} &\quad&\text{if } g_p(v,v) > 0. \label{eq:spacelike}
\end{alignat}
\end{definition}

\begin{definition}[Null Cone]\label{def:null-cone}
The null cone at $p$ is the codimension-1 hypersurface in $T_pM$:
\begin{equation}\label{eq:null-cone}
  \mathcal{C}_p := \bigl\{v \in T_pM \setminus \{0\} : g_p(v,v) = 0\bigr\}.
\end{equation}
A time orientation selects the \emph{future} sheet $\mathcal{C}_p^+$. The
future light cone in spacetime is $\partial J^+(p)$.
\end{definition}

\subsection{The Speed of Light as Conformal Invariant}

\begin{definition}[Speed of Causal Propagation]\label{def:speed-c}
In a coordinate system $(x^0, x^1, \ldots, x^{n-1})$ adapted to the
conformal structure, define:
\begin{equation}\label{eq:c-def}
  c := \sup_{\substack{p \in M \\ \gamma \text{ causal}}}
    \frac{d_{\text{space}}(\gamma)}{d_{\text{time}}(\gamma)},
\end{equation}
where $d_{\text{space}}$ and $d_{\text{time}}$ are the spatial and temporal
arc-length functionals induced by the $(-,+,\ldots,+)$ decomposition.
\end{definition}

\begin{proposition}[Conformal Invariance of $c$]\label{prop:c-conformal}
Let $\tilde{g} = \Lambda^2 g$ for smooth $\Lambda > 0$. The null condition
$g_p(v,v) = 0$ is equivalent to $\tilde{g}_p(v,v) = 0$. Hence the null
cone $\mathcal{C}_p$---and therefore $c$---is a conformal invariant.
\end{proposition}

\begin{proof}
$\tilde{g}_p(v,v) = \Lambda^2(p)\, g_p(v,v) = 0$ if and only if $g_p(v,v) = 0$
(since $\Lambda(p) > 0$).
\end{proof}

\begin{remark}[Naturalness of $c = 1$]\label{rem:c-natural}
In geometric (natural) units, we set $c = 1$: the null cone has unit slope,
and the metric takes the form $\eta = \text{diag}(-1, +1, \ldots, +1)$ in
Minkowski space. The SI value $c = 299\,792\,458\;\text{m/s}$ reflects a
specific choice of anthropogenic units (see \Cref{sec:anthropogenic-c}).
\end{remark}

\section{The Causal Hierarchy}\label{sec:hierarchy}

We state the standard causality conditions, emphasizing their relation
to the TONLEE framework.

\begin{definition}[Causality Conditions]\label{def:causality-hierarchy}
A time-oriented Lorentzian manifold $(M,g)$ is:
\begin{enumerate}[label=(\arabic*), leftmargin=3em]
  \item \textbf{Chronological} if $p \notin I^+(p)$ for all $p \in M$
    (no closed timelike curves).
  \item \textbf{Causal} if $\preceq$ is antisymmetric
    (no closed causal curves), i.e., $\preceq$ is a partial order.
  \item \textbf{Distinguishing} if $I^+(p) = I^+(q) \Rightarrow p = q$
    and $I^-(p) = I^-(q) \Rightarrow p = q$.
  \item \textbf{Strongly causal} if the Alexandrov topology coincides with
    the manifold topology.
  \item \textbf{Stably causal} if there exists a temporal function
    $t: M \to \R$ with $p \prec q \Rightarrow t(p) < t(q)$.
  \item \textbf{Globally hyperbolic} if it is causal and for all $p, q \in M$,
    the causal diamond $J(p,q) = J^+(p) \cap J^-(q)$ is compact.
\end{enumerate}
These form a strict hierarchy:
\begin{equation}\label{eq:hierarchy}
  \text{(6)} \;\Longrightarrow\; \text{(5)} \;\Longrightarrow\; \text{(4)}
  \;\Longrightarrow\; \text{(3)} \;\Longrightarrow\; \text{(2)}
  \;\Longrightarrow\; \text{(1)}.
\end{equation}
\end{definition}

\begin{theorem}[Geroch 1970; Bernal--S\'{a}nchez 2003]\label{thm:geroch}
$(M, g)$ is globally hyperbolic if and only if it admits a \emph{Cauchy surface}
$\Sigma$---a subset intersected exactly once by every inextendible timelike curve.
Moreover, $M$ is diffeomorphic to $\R \times \Sigma$, and
$g = -\beta^2\, dt^2 + h_t$, where $h_t$ is a Riemannian metric on
each $\{t\} \times \Sigma$ and $\beta > 0$ is the lapse function.
\end{theorem}


%======================================================================
% PART III
%======================================================================
\part{The Ignorance Functional and Emergent Dynamics}

\section{Construction of the Ignorance Functional}\label{sec:ignorance}

\subsection{Motivation: Causality as Residual Mismatch}

In the TONLEE framework, physical dynamics are driven by the universe's
tendency to minimize residual mismatch---to approach the null state.
We formalize this through a variational principle.

\begin{definition}[Field Content]\label{def:field-content}
On a globally hyperbolic spacetime $(M, g)$ with $M \cong \R \times \Sigma$,
define the \emph{mismatch field configuration} as a pair $(m, \psi)$ where:
\begin{itemize}[leftmargin=2em]
  \item $m: M \to \R$ is the \emph{gross field} (a real scalar field
    representing mass-energy mismatch),
  \item $\psi: M \to \C$ is the \emph{subtle field} (a complex scalar field
    representing quantum-informational mismatch).
\end{itemize}
Both fields are assumed smooth and of compact spatial support on each slice $\Sigma_t$.
\end{definition}

\begin{definition}[Ignorance Functional]\label{def:ignorance-functional}
The \emph{ignorance functional} is the action:
\begin{equation}\label{eq:ignorance}
  \II[m, \psi; g] = \int_M \bigl(\Lag_{\text{gross}} + \Lag_{\text{subtle}}
    + \Lag_{\text{coupling}}\bigr)\sqrt{-\det g}\; d^nx,
\end{equation}
where:
\begin{align}
  \Lag_{\text{gross}} &= \frac{1}{2}\,g^{\mu\nu}\partial_\mu m\,\partial_\nu m
    - V(m), \label{eq:L-gross}\\
  \Lag_{\text{subtle}} &= \frac{i\hbar}{2}\bigl(\psi^*\partial_t\psi
    - (\partial_t\psi^*)\psi\bigr)
    - \frac{\hbar^2}{2\mu_{\text{eff}}}\,g^{ij}\partial_i\psi^*\partial_j\psi,
    \label{eq:L-subtle}\\
  \Lag_{\text{coupling}} &= \lambda\, m\,|\psi|^2, \label{eq:L-coupling}
\end{align}
with $V: \R \to \R_{\geq 0}$ a potential satisfying $V(0) = 0$ and $V''(0) \geq 0$,
$\mu_{\text{eff}} > 0$ an effective inertial parameter, and $\lambda \in \R$ a
coupling constant.
\end{definition}

\begin{remark}
The specific form of $\Lag_{\text{subtle}}$ in \eqref{eq:L-subtle} is that of
a Schr\"{o}dinger field on a curved background---the non-relativistic limit of
a complex Klein--Gordon field. This is deliberate: in the TONLEE picture, the
quantum (subtle) sector operates on the simultaneity surfaces $\Sigma_t$, which
are intrinsically Riemannian. The Schr\"{o}dinger structure is \emph{inherited}
from the Lorentzian ambient space, not postulated independently.
\end{remark}

\subsection{The Nullity Constraint}

\begin{definition}[Global Nullity Constraint]\label{def:nullity-constraint}
The field configuration $(m, \psi)$ must satisfy the global constraint:
\begin{equation}\label{eq:nullity-constraint}
  \int_M \bigl(T_{\mu\nu}^{(\text{gross})} + T_{\mu\nu}^{(\text{subtle})}\bigr)
  \,n^\mu\, d\Sigma^\nu = 0,
\end{equation}
for every Cauchy surface $\Sigma$, where $T_{\mu\nu}$ is the total
stress-energy tensor and $n^\mu$ is the future-directed unit normal to $\Sigma$.
This is the field-theoretic expression of Axiom~\ref{ax:nullity}: the total
energy on any simultaneity surface is zero.
\end{definition}

\subsection{The Minimization Principle}

\begin{axiom}[Causality Minimization]\label{ax:minimization}
The physical field configuration $(m_*, \psi_*)$ extremizes the ignorance
functional subject to the nullity constraint:
\begin{equation}\label{eq:extremize}
  \delta\II[m, \psi; g] = 0 \quad\text{subject to}\quad
  \int_\Sigma \bigl(\mathcal{E}_{\text{gross}} + \mathcal{E}_{\text{subtle}}\bigr)\,d\mu_\Sigma = 0,
\end{equation}
where $\mathcal{E}$ denotes the energy density.
\end{axiom}

\section{Euler--Lagrange Equations}\label{sec:EL}

\begin{theorem}[Equations of Motion]\label{thm:EL}
The Euler--Lagrange equations of the ignorance functional \eqref{eq:ignorance}
are:
\begin{align}
  \text{\emph{Gross sector:}} \qquad
    \Box_g m + V'(m) + \lambda|\psi|^2 &= 0,
    \label{eq:EL-gross}\\[4pt]
  \text{\emph{Subtle sector:}} \qquad
    i\hbar\,\partial_t\psi + \frac{\hbar^2}{2\mu_{\emph{eff}}}\Delta_\Sigma\psi
    - \lambda\, m\,\psi &= 0,
    \label{eq:EL-subtle}
\end{align}
where $\Box_g = g^{\mu\nu}\nabla_\mu\nabla_\nu$ is the d'Alembertian on $(M,g)$
and $\Delta_\Sigma = h^{ij}\nabla_i\nabla_j$ is the Laplace--Beltrami operator
on the Cauchy surface $(\Sigma, h)$.
\end{theorem}

\begin{proof}
For the gross sector, compute $\frac{\delta\II}{\delta m} = 0$:
\begin{align}
  \frac{\delta}{\delta m}\int_M \Bigl(\frac{1}{2}g^{\mu\nu}\partial_\mu m\,\partial_\nu m
    - V(m) + \lambda m|\psi|^2\Bigr)\sqrt{-g}\,d^nx &= 0 \nonumber\\
  \Longrightarrow\quad -\Box_g m - V'(m) - \lambda|\psi|^2 &= 0.
\end{align}
Here we used the standard identity for the variation of the kinetic term:
$\delta\bigl(\frac{1}{2}g^{\mu\nu}\partial_\mu m\,\partial_\nu m\,\sqrt{-g}\bigr)
= -(\Box_g m)\,\delta m\,\sqrt{-g} + \text{boundary terms}$.

For the subtle sector, compute $\frac{\delta\II}{\delta\psi^*} = 0$. The
$\Lag_{\text{subtle}}$ Lagrangian yields a Schr\"{o}dinger-type equation on
$(\Sigma, h)$ with potential $\lambda m$:
\begin{equation}
  i\hbar\,\partial_t\psi = -\frac{\hbar^2}{2\mu_{\text{eff}}}\Delta_\Sigma\psi
    + \lambda\, m\,\psi. \qedhere
\end{equation}
\end{proof}

\section{Emergence of the Speed of Light}\label{sec:c-derivation}

\subsection{Well-Posedness Requires Lorentzian Signature}

\begin{theorem}[Causal Speed from Variational Stability]\label{thm:c-emerge}
For the ignorance functional \eqref{eq:ignorance} to have:
\begin{enumerate}[label=(\alph*)]
  \item a well-defined Cauchy problem (existence, uniqueness, and continuous
    dependence on initial data), and
  \item finite propagation speed (compact causal support of solutions),
\end{enumerate}
the metric $g$ must be of Lorentzian signature $(-,+,\ldots,+)$, and the
gross-sector equation \eqref{eq:EL-gross} must be hyperbolic. The maximal
propagation speed is:
\begin{equation}\label{eq:c-from-metric}
  c^2 = \frac{|g^{00}|}{g^{ii}} \quad (\text{no sum}),
\end{equation}
evaluated in an orthonormal frame. In a maximally symmetric (homogeneous,
isotropic) background---the state closest to nullity---this ratio is a
universal constant.
\end{theorem}

\begin{proof}[Proof Strategy]
Consider the principal symbol of the operator $\Box_g + V''(m_0)$ linearized
about a background $m_0$:
\begin{equation}
  \sigma_{\text{princ}}(\xi) = g^{\mu\nu}\xi_\mu\xi_\nu.
\end{equation}
The equation is hyperbolic if and only if the principal symbol has Lorentzian
signature. The characteristic surfaces (wavefronts) satisfy
$g^{\mu\nu}\xi_\mu\xi_\nu = 0$, i.e., $\xi$ is null. By the theory of
hyperbolic PDEs (Leray, 1953; H\"{o}rmander, 1963), solutions to the
Cauchy problem have support in $J^+(\supp(\text{data}))$---the causal future
of the initial data support. The propagation speed along any spatial direction
$e_i$ is bounded by $c = \sqrt{|g^{00}|/g^{ii}}$ in the chosen frame.

In particular, if $g$ were Riemannian (all positive eigenvalues), the equation
would be elliptic, the Cauchy problem would be ill-posed (Hadamard), and no
finite propagation speed would exist. Lorentzian signature is thus not a
free choice but a \emph{consequence} of requiring the ignorance functional
to have stable, causal extrema.
\end{proof}

\subsection{The Null Geodesic as Zero-Mismatch Trajectory}

\begin{proposition}\label{prop:null-geodesic}
Setting $m \equiv 0$ and $\psi \equiv 0$ in the gross-sector equation
\eqref{eq:EL-gross} yields the trivially satisfied identity $0 = 0$
(``absolute nullity''). The first nontrivial solutions are the
\emph{massless modes}: linearizing about $m_0 = 0$ with $V'(0) = 0$
and $\lambda = 0$:
\begin{equation}\label{eq:massless-wave}
  \Box_g\, \delta m = 0.
\end{equation}
The characteristics of \eqref{eq:massless-wave} satisfy
$g^{\mu\nu}k_\mu k_\nu = 0$---the null geodesic equation for the
wave-covector $k_\mu$. In the geometric optics limit, the rays are
null geodesics of $(M,g)$, propagating at speed $c$.
\end{proposition}

\begin{remark}
The photon---the massless excitation---does not ``travel at the speed of
light.'' Rather, $c$ is \emph{defined by} the null cone of $(M,g)$, which
is the characteristic surface of the wave equation arising from the ignorance
functional. The photon traces the boundary between the causal and acausal
domains of spacetime.
\end{remark}

\subsection{Anthropogenic Units and the Value $c = 299\,792\,458\;\text{m/s}$}
\label{sec:anthropogenic-c}

\begin{proposition}[Unit Analysis]\label{prop:anthropogenic}
Since 1983, the meter is defined as:
\begin{equation}
  1\;\text{m} := \frac{c}{299\,792\,458}\;\text{light-seconds},
\end{equation}
making the numerical value of $c$ in SI tautological. The non-trivial
content is the \emph{historical coincidence}: the original meter
($\approx 10^{-7}$ of the Earth's quadrant) and the second ($1/86\,400$ of
a solar day, later redefined via the cesium-133 hyperfine transition at
$\Delta\nu_{\text{Cs}} = 9\,192\,631\,770\;\text{Hz}$) produce the ratio:
\begin{equation}\label{eq:anthropogenic-ratio}
  c = \frac{\text{conformal conversion factor}}{\text{(anthropogenic length)}/
  \text{(anthropogenic time)}}
  = \frac{1\;\text{[geometric]}}{1\;\text{m}/1\;\text{s}}
  = 299\,792\,458\;\frac{\text{m}}{\text{s}}.
\end{equation}
This ratio encodes the hierarchy of mismatch scales: the electromagnetic
scale ($\Delta\nu_{\text{Cs}}$) and the gravitational scale
($C_\oplus \approx 4 \times 10^7\;\text{m}$), linked by:
\begin{equation}\label{eq:scale-hierarchy}
  \frac{c}{C_\oplus \cdot \Delta\nu_{\text{Cs}}}
  \approx 8.1 \times 10^{-10},
\end{equation}
a dimensionless number determined by $\alpha$, $G$, and the particle masses.
\end{proposition}


%======================================================================
% PART IV
%======================================================================
\part{Simultaneity, Quantum Mechanics, and Black Holes}

\section{Simultaneity Surfaces and Entanglement}\label{sec:simultaneity}

\subsection{Spacelike Hypersurfaces as Quantum Arenas}

\begin{definition}[Simultaneity Surface]\label{def:simultaneity}
A \emph{simultaneity surface} in $(M, g)$ is a smooth, achronal, spacelike
hypersurface $\Sigma \subset M$:
\begin{equation}\label{eq:spacelike-hyp}
  g_p(n, n) < 0 \qquad \forall\, p \in \Sigma,
\end{equation}
where $n$ is the unit normal to $\Sigma$. (Convention: $n$ is timelike, hence
$g(n,n) < 0$; vectors tangent to $\Sigma$ are spacelike.)
\end{definition}

\begin{postulate}[Simultaneity $\Rightarrow$ Quantum Correlations]\label{post:sim-QM}
Two events $p, q \in M$ can share quantum correlations (entanglement) only if:
\begin{enumerate}[label=(\roman*)]
  \item There exists a simultaneity surface $\Sigma$ with $p, q \in \Sigma$.
  \item There exists a common causal ancestor: $\exists\, r \in M$ with
    $r \prec p$ and $r \prec q$ (i.e., $p, q \in J^+(r)$).
\end{enumerate}
\end{postulate}

\begin{remark}
This postulate does not contradict the standard formulation of quantum
mechanics on a fixed background: any pair of spacelike-separated events
that are entangled in the standard formulation always satisfies conditions~(i)
and~(ii) above, because in a globally hyperbolic spacetime, every pair of
causally related events to a common past share a Cauchy surface. The postulate
becomes nontrivial in the black hole interior where simultaneity surfaces
degenerate (see \Cref{sec:BH-interior}).
\end{remark}

\subsection{Microcausality from the Ignorance Functional}

\begin{theorem}[Emergent Microcausality]\label{thm:microcausality}
Let $\hat{\phi}(x)$ be the quantized gross field satisfying the linearized
equation $\Box_g \hat{\phi} = 0$ on a globally hyperbolic $(M, g)$. Define
the causal propagator:
\begin{equation}
  \Delta(x, y) := G_{\text{ret}}(x, y) - G_{\text{adv}}(x, y),
\end{equation}
where $G_{\text{ret/adv}}$ are the retarded/advanced Green's functions of
$\Box_g$. Then:
\begin{enumerate}[label=(\alph*)]
  \item $\supp\bigl(\Delta(\cdot, y)\bigr) \subseteq J^+(y) \cup J^-(y)$.
  \item The canonical commutation relation is:
    \begin{equation}\label{eq:CCR}
      [\hat{\phi}(x), \hat{\phi}(y)] = i\,\Delta(x,y)\,\hat{\mathbb{1}}.
    \end{equation}
  \item For spacelike-separated $x, y$ (i.e., $x \notin J^+(y) \cup J^-(y)$):
    \begin{equation}\label{eq:microcausal}
      [\hat{\phi}(x), \hat{\phi}(y)] = 0.
    \end{equation}
\end{enumerate}
\end{theorem}

\begin{proof}
Part~(a) follows from the support properties of retarded/advanced Green's
functions on globally hyperbolic spacetimes (Leray, 1953; B\"{a}r--Ginoux--Pf\"{a}ffle,
2007). Part~(b) is the Peierls bracket construction: the commutator of the
free field is fixed by the causal propagator. Part~(c) is immediate from
(a) and (b): if $x$ and $y$ are spacelike separated, then
$x \notin J^+(y) \cup J^-(y)$, so $\Delta(x,y) = 0$.
\end{proof}

\section{Black Hole Interiors: Simultaneity Breakdown}\label{sec:BH-interior}

\subsection{Coordinate Signature Inversion}

\begin{theorem}[Simultaneity Failure in the Schwarzschild Interior]\label{thm:BH-sim}
In the Schwarzschild spacetime with mass parameter $M_{\text{BH}}$,
the metric in Schwarzschild coordinates $(t, r, \theta, \varphi)$ is:
\begin{equation}\label{eq:schwarzschild}
  ds^2 = -\left(1 - \frac{r_s}{r}\right)c^2\,dt^2
  + \left(1 - \frac{r_s}{r}\right)^{-1}dr^2 + r^2\,d\Omega^2,
\end{equation}
where $r_s = 2GM_{\text{BH}}/c^2$. For $r < r_s$:
\begin{enumerate}[label=(\roman*)]
  \item The coefficient of $dt^2$ becomes positive and the coefficient of
    $dr^2$ becomes negative: $r$ is now a timelike coordinate and $t$ is
    spacelike.
  \item A surface of constant $r < r_s$ has induced metric of signature
    $(+, +, +)$ on $\{t, \theta, \varphi\}$---it is \emph{spacelike}
    only in the $\{t\}$-direction but $r = \text{const}$ surfaces are
    actually timelike (they contain the timelike direction $\partial_r$
    restricted to the surface is not tangent; rather, $\partial_t$ is
    now spacelike and is tangent to the surface).
\end{enumerate}
More precisely: for $r < r_s$, the gradient $dr$ has
$g^{rr} = -(1 - r_s/r) > 0$, so $dr$ is a \emph{timelike} 1-form.
Surfaces of constant $r$ have $dr = 0$ as their defining condition, so they
are level sets of a timelike function---hence they are \emph{spacelike}
hypersurfaces. However, two events at the same $r < r_s$ but
different angular positions are connected by a spacelike geodesic on this
surface, while events at \emph{different} $r$-values inside the horizon are
necessarily separated by a timelike interval (since $r$ is the time direction).
This means that an observer falling inward cannot construct a simultaneity
surface connecting events at different stages of their radial infall---the
``simultaneity surfaces'' shrink as one approaches $r = 0$.
\end{theorem}

\begin{corollary}[TONLEE Prediction: Quantum Decoherence Near Horizons]\label{cor:decoherence}
Since \Cref{post:sim-QM} requires a simultaneity surface for entanglement,
and the available simultaneity surfaces shrink (in spatial extent) as
$r \to r_s^+$ from outside and as $r \to 0$ from inside, the TONLEE
framework predicts:
\begin{equation}\label{eq:decoherence-rate}
  \tau_{\text{coherence}} \;\propto\; \text{Area}(\Sigma \cap \mathcal{U}),
\end{equation}
where $\mathcal{U}$ is the observer's causal neighborhood. Near the horizon,
$\text{Area}(\Sigma \cap \mathcal{U}) \to 0$, and quantum coherence times
should measurably decrease.
\end{corollary}


%======================================================================
% PART V
%======================================================================
\part{The Fibonacci--Ramanujan Null Universe}

\section{The Fibonacci Generating Function and Regularization}\label{sec:fibonacci}

\subsection{The Generating Function}

\begin{definition}
The Fibonacci sequence $\{F_n\}_{n \geq 0}$ is defined by $F_0 = 0$, $F_1 = 1$,
and $F_{n+2} = F_{n+1} + F_n$. Its ordinary generating function is:
\begin{equation}\label{eq:fib-ogf}
  f(x) = \sum_{n=0}^\infty F_n\, x^n = \frac{x}{1 - x - x^2},
\end{equation}
convergent for $|x| < 1/\varphi$, where $\varphi = \frac{1+\sqrt{5}}{2}$
is the golden ratio.
\end{definition}

\subsection{Analytic Continuation and the Value $f(1) = -1$}

\begin{theorem}[Fibonacci--Ramanujan Identity]\label{thm:fib-ram}
The rational function $f(x) = \frac{x}{1 - x - x^2}$ has a meromorphic
extension to all of $\C$, with simple poles at $x = -\varphi$ and
$x = 1/\varphi = \varphi - 1$. At $x = 1$ (which lies between the two poles
and is not a singularity of $f$... let us verify):

\emph{Correction:} The denominator $1 - x - x^2 = -(x^2 + x - 1)$ vanishes at
$x = \frac{-1 \pm \sqrt{5}}{2}$, i.e., at $x = \varphi - 1 \approx 0.618$
and $x = -\varphi \approx -1.618$.

Since $x = 1$ lies \emph{outside} the radius of convergence
($1 > 1/\varphi \approx 0.618$) but is not a pole of $f$, the value
$f(1)$ is well-defined as a value of the rational function:
\begin{equation}\label{eq:fib-reg}
  f(1) = \frac{1}{1 - 1 - 1} = \frac{1}{-1} = -1.
\end{equation}
This serves as a regularized value for the formal divergent series
$\sum_{n=0}^\infty F_n$.
\end{theorem}

\begin{proof}
The generating function $f(x) = x/(1 - x - x^2)$ is a rational function
with poles at $x_{\pm} = (-1 \pm \sqrt{5})/2$. Since $x = 1$ is neither
of these values ($x_+ \approx 0.618$, $x_- \approx -1.618$), the rational
function is well-defined at $x = 1$. The computation is direct substitution.

The interpretation as a ``regularized sum'' follows from the general
principle: for a power series $\sum a_n x^n$ with finite radius of convergence
$R$, if the function defined by the series has an analytic (or meromorphic)
continuation beyond $R$, the value at a point $x_0 > R$ (provided $x_0$
is not a pole) serves as a canonical regularized value for $\sum a_n x_0^n$.
This is precisely the Abel summation philosophy, and coincides with
Ramanujan's summation method for rational generating functions.
\end{proof}

\begin{remark}[Subtlety: $x = 1$ is Beyond a Pole]
Since the pole at $x_+ = (\sqrt{5}-1)/2 \approx 0.618$ lies between $0$ and $1$,
the analytic continuation from $|x| < 1/\varphi$ to $x = 1$ must ``pass through''
this singularity. The rational function $f(x)$ is of course defined at $x = 1$
regardless of the power series, but the identification of $f(1)$ with the
regularized value of $\sum F_n$ is a statement about the rational function,
not about summability in the Abel or Ces\`{a}ro sense.

Technically, $\sum F_n x^n$ is \emph{not} Abel summable (the limit
$\lim_{x \to 1^-} f(x)$ does not exist because of the pole at $x_+ < 1$).
The value $f(1) = -1$ should therefore be understood as a
\emph{rational-function regularization}: the unique value assigned to the
formal series by its generating function, evaluated at $x = 1$ in the
rational (not power-series) sense.
\end{remark}

\subsection{The Null Universe Identity}

\begin{definition}[Unit Universe and Its Reflection]\label{def:unit-universe}
Define:
\begin{align}
  U_+ &:= +1 \quad\text{(the unit universe: total positive mismatch)}, \\
  U_- &:= -1 \quad\text{(the holographic reflection: the Fibonacci--Ramanujan shadow)}.
\end{align}
\end{definition}

\begin{proposition}[Null Identity]\label{prop:null-identity}
\begin{equation}\label{eq:null-identity}
  U_+ + U_- = +1 + (-1) = 0.
\end{equation}
This is the arithmetic instantiation of \Cref{ax:nullity}.
\end{proposition}

\subsection{The Golden Ratio as Mismatch Growth Rate}

\begin{proposition}\label{prop:golden-ratio}
The Fibonacci recurrence $F_{n+2} = F_{n+1} + F_n$ has the asymptotic
solution $F_n \sim \varphi^n / \sqrt{5}$. The golden ratio
$\varphi = (1 + \sqrt{5})/2$ satisfies:
\begin{equation}\label{eq:golden-fixed}
  \varphi^2 = \varphi + 1, \qquad \varphi = 1 + \cfrac{1}{1 + \cfrac{1}{1 + \cfrac{1}{\ddots}}}.
\end{equation}
In the TONLEE interpretation, $\varphi$ encodes the discrete mismatch
growth rate: each new mismatch level is generated by the combination of
the two preceding levels, and the ratio of consecutive levels converges
to $\varphi$.
\end{proposition}


\section{Dark Sector as Intrinsic/Extrinsic Mismatch Decomposition}\label{sec:dark}

\subsection{The Two Pathways to Nullity}

\begin{definition}[Dark Sector Decomposition]\label{def:dark-sector}
Decompose the ignorance functional into intrinsic and extrinsic components:
\begin{equation}
  \II = \II_{\text{int}} + \II_{\text{ext}},
\end{equation}
where:
\begin{itemize}[leftmargin=2em]
  \item $\II_{\text{int}}$ contains terms that drive $m \to 0$ locally
    (gravitational collapse, annihilation---the ``intrinsic pathway'').
  \item $\II_{\text{ext}}$ contains terms that drive $\psi \to 0$ globally
    (expansion, dilution---the ``extrinsic pathway'').
\end{itemize}
Define the effective weights:
\begin{equation}\label{eq:dark-weights}
  w_{\text{DM}} := \frac{\delta\II_{\text{int}}/\delta m}{\delta\II/\delta m},
  \qquad
  w_{\text{DE}} := \frac{\delta\II_{\text{ext}}/\delta\psi^*}{\delta\II/\delta\psi^*}.
\end{equation}
\end{definition}

\begin{postulate}[Fibonacci Equilibrium]\label{post:fib-eq}
At cosmological equilibrium (the state of maximally efficient approach
to nullity), the dark-sector ratio attains:
\begin{equation}\label{eq:fib-meeting}
  \frac{w_{\text{DM}}}{w_{\text{DE}}} = \varphi^{-1} = \varphi - 1
  \approx 0.618.
\end{equation}
\end{postulate}

\begin{remark}[Observational Status]\label{rem:observed-ratio}
Current observations give $\Omega_{\text{DM}}/\Omega_{\text{DE}} \approx
0.27/0.68 \approx 0.397$. Including baryonic matter:
$(\Omega_{\text{DM}} + \Omega_b)/\Omega_{\text{DE}} \approx 0.32/0.68
\approx 0.47$. The discrepancy from $\varphi^{-1} \approx 0.618$ may indicate
the universe has not yet reached the Fibonacci equilibrium, that baryonic matter
represents an additional mismatch component not yet fully resolved, or that
the identification \eqref{eq:fib-meeting} requires modification. This remains
an open problem.
\end{remark}

\subsection{Cosmological Nullity}

\begin{proposition}[Total Dark Sector Cancellation]\label{prop:cosmo-null}
In the TONLEE framework, the total cosmological energy budget satisfies:
\begin{equation}\label{eq:cosmo-null}
  E_{\text{DM}} + E_{\text{DE}} + E_{\text{visible}}
  + E_{\text{gravitational}} = 0.
\end{equation}
This is consistent with the standard result that in a spatially flat
Friedmann universe ($k = 0$), the total energy (kinetic + potential +
matter) vanishes identically. In the Friedmann equation:
\begin{equation}
  H^2 = \frac{8\pi G}{3}\rho_{\text{total}} - \frac{k}{a^2},
\end{equation}
setting $k = 0$ does not by itself enforce $E_{\text{total}} = 0$, but the
Newtonian energy argument (Tryon, 1973) shows that for a flat universe,
the gravitational binding energy exactly compensates the mass-energy content,
yielding $E_{\text{total}} = 0$---consistent with \Cref{ax:nullity}.
\end{proposition}


%======================================================================
% PART VI
%======================================================================
\part{Information, Domain of Dependence, and Predictions}

\section{Information as Causal Cardinality}\label{sec:information}

\begin{definition}[Information Content]\label{def:info}
For a spacetime region $D \subseteq M$, define the \emph{information content}
as the cardinality of the causal relation restricted to $D$:
\begin{equation}\label{eq:info}
  I(D) := \bigl|\{(p,q) \in D \times D : p \prec q\}\bigr|.
\end{equation}
In the continuum limit (where $D$ contains $N$ causal-set elements), this
scales as:
\begin{equation}
  I(D) \sim C_n \cdot N^2,
\end{equation}
where $C_n$ is a dimension-dependent constant encoding the average
connectivity of the causal set in $n$ dimensions.
\end{definition}

\begin{theorem}[Domain of Dependence and Causal Determinism]\label{thm:DoD}
Let $S \subset M$ be a closed achronal set in a globally hyperbolic spacetime.
The future domain of dependence is:
\begin{equation}
  D^+(S) := \bigl\{p \in M : \text{every past-inextendible causal curve through }
  p \text{ intersects } S\bigr\}.
\end{equation}
For a well-posed hyperbolic system (such as equation \eqref{eq:EL-gross}),
the solution in $D^+(S)$ is uniquely determined by data on $S$. Moreover:
\begin{equation}
  \supp(\text{solution}) \subseteq J^+\bigl(\supp(\text{data})\bigr).
\end{equation}
This is the PDE-theoretic expression of causality: information (in the sense
of \Cref{def:info}) cannot propagate outside the light cone.
\end{theorem}

\section{Black Holes as Causality Destroyers}\label{sec:BH-causality}

\begin{definition}[Black Hole Region]\label{def:BH}
In an asymptotically flat spacetime $(M, g)$ with conformal boundary
$\mathscr{I}^+$ (future null infinity), the \emph{black hole region} is:
\begin{equation}\label{eq:BH-region}
  \mathcal{B} := M \setminus J^-(\mathscr{I}^+),
\end{equation}
and the \emph{event horizon} is:
\begin{equation}\label{eq:horizon}
  \mathcal{H}^+ := \partial J^-(\mathscr{I}^+) \cap M.
\end{equation}
\end{definition}

\begin{theorem}[Hawking's Area Theorem, 1971]\label{thm:area}
If $(M, g)$ satisfies the null energy condition
$R_{\mu\nu}k^\mu k^\nu \geq 0$ for all null $k^\mu$, and if the spacetime
is future asymptotically predictable, then the area of the event horizon
$\mathcal{H}^+$ is non-decreasing along any future-directed null generator:
\begin{equation}\label{eq:area-theorem}
  \frac{dA}{d\lambda} \geq 0,
\end{equation}
where $\lambda$ is the affine parameter along the generators.
\end{theorem}

\begin{remark}[TONLEE Reinterpretation]
The area theorem is standardly interpreted as black hole entropy
($S_{\text{BH}} = k_B A / 4\ell_P^2$) increasing. In the TONLEE framework,
the increase in horizon area represents the \emph{destruction of causal
structure}: causally connected spacetime is converted into the causally
disconnected interior. Information content $I$ decreases outside the horizon,
registering as an entropy increase (loss of accessible causal relations).
The TONLEE prediction is that isolated black holes expand (horizon area
increases) even without infalling matter, reinterpreting Hawking radiation
as the mechanism of this causal destruction.
\end{remark}

\section{Summary of Physical Predictions}\label{sec:predictions}

The TONLEE framework generates the following testable predictions, stated
with their mathematical origins:

\begin{enumerate}[label=\textbf{P\arabic*.}, leftmargin=3em]
  \item \textbf{No gravitons.} Gravity is the conformal arena (\Cref{thm:Malament}),
    not a quantum field. Quantum gravity scattering cross-sections are predicted
    to be exactly zero.

  \item \textbf{Dark matter is not a particle.} Direct detection experiments
    should yield null results. Dark matter effects are encoded in the causal
    memory structure (\Cref{def:dark-sector}).

  \item \textbf{Black holes expand, not evaporate.} The horizon area strictly
    increases (\Cref{thm:area}), reinterpreted as causal destruction.

  \item \textbf{Gravitational decoherence.} Quantum coherence times decrease
    in strong gravitational fields (\Cref{cor:decoherence}), scaling with the
    area of available simultaneity surfaces.

  \item \textbf{Entanglement requires shared causal past.} Entanglement is
    impossible between particles whose entire histories lie inside a black hole
    (\Cref{post:sim-QM} + \Cref{thm:BH-sim}).

  \item \textbf{No initial singularity.} The CMB should be consistent with a
    no-boundary condition rather than a sharp initial singularity
    (\Cref{ax:nullity} applied cosmologically).

  \item \textbf{Cosmological constant vanishes in total.} $\Lambda_{\text{obs}} > 0$
    is a local mismatch; $\int_{\text{total}} \Lambda\,d\mu = 0$
    (\Cref{prop:cosmo-null}).
\end{enumerate}


%======================================================================
% APPENDIX
%======================================================================
\appendix

\section{Algebraic Quantum Field Theory: Haag--Kastler Axioms}\label{app:AQFT}

For completeness, we state the standard axiomatic framework that encodes
causality in quantum field theory.

\begin{definition}[Haag--Kastler Net]\label{def:HK}
A \emph{local net of observables} on $(M, g)$ is an assignment
$\mathcal{O} \mapsto \Alg(\mathcal{O})$ from open, relatively compact
regions $\mathcal{O} \subset M$ to unital $C^*$-algebras, satisfying:
\begin{enumerate}[label=(A\arabic*)]
  \item \textbf{Isotony:} $\mathcal{O}_1 \subseteq \mathcal{O}_2 \Rightarrow
    \Alg(\mathcal{O}_1) \subseteq \Alg(\mathcal{O}_2)$.
  \item \textbf{Microcausality:} If $\mathcal{O}_1 \perp \mathcal{O}_2$
    (spacelike separated), then
    $[A, B] = 0$ for all $A \in \Alg(\mathcal{O}_1)$, $B \in \Alg(\mathcal{O}_2)$.
  \item \textbf{Covariance:} There exists a representation $\alpha$ of the
    isometry group by automorphisms: $\alpha_g(\Alg(\mathcal{O})) = \Alg(g \cdot \mathcal{O})$.
  \item \textbf{Spectrum condition:} In the vacuum representation, the joint
    spectrum of the translation generators $P^\mu$ lies in $\overline{V^+}$
    (closed forward light cone).
\end{enumerate}
\end{definition}

The connection to TONLEE is direct: axiom (A2) is the operator-algebraic
expression of the causal structure established in \Cref{sec:causal-order},
and \Cref{thm:microcausality} shows that it is derivable from the ignorance
functional's hyperbolic structure.


\section{Process Matrices and Indefinite Causal Order}\label{app:process}

\begin{definition}[Process Matrix]\label{def:process}
For $N$ parties with input/output Hilbert spaces $\Hil^{A_i^I} \otimes \Hil^{A_i^O}$,
a \emph{process matrix} is:
\begin{equation}
  W \in \Lag\!\left(\bigotimes_{i=1}^N \Hil^{A_i^I} \otimes \Hil^{A_i^O}\right),
  \qquad W \geq 0,
\end{equation}
satisfying normalization: $\Tr\bigl[W(\bigotimes_i M^{A_i})\bigr] = 1$ for
all CPTP maps $\{M^{A_i}\}$.
\end{definition}

\begin{definition}[Causal Separability]\label{def:causal-sep}
$W$ is \emph{causally separable} if:
\begin{equation}
  W = \sum_\sigma q_\sigma\, W_\sigma, \qquad q_\sigma \geq 0,\;
  \sum_\sigma q_\sigma = 1,
\end{equation}
where each $W_\sigma$ is compatible with a definite causal order $\sigma$.
A violation of causal separability witnesses \emph{indefinite causal order}.
\end{definition}

\begin{remark}
In TONLEE, a causally inseparable process matrix $W$ represents a mismatch
configuration in which the causal order itself is ``mismatched''---the
partial order $\preceq$ is not globally well-defined. Such configurations
are expected near black hole horizons where the causal structure degrades
(\Cref{thm:BH-sim}).
\end{remark}

\bigskip
\begin{center}
\rule{0.5\textwidth}{0.4pt}
\end{center}
\bigskip

\begin{quote}
\emph{Nothing is the hardest thing to understand, because it is the only
thing that needs no explanation---and yet explains everything.}
\hfill---TONLEE
\end{quote}

\end{document}
